\begin{table}[h]
\centering
\begin{tabular}{cl}
\toprule
\textbf{Symbol} & \textbf{Description} \\
\midrule
$\Omega$ & Sample space \\
$E$ & Event \\
$Pr$ & Probability distribution \\
$Pr[A \mid B]$ & Probability of event A given event B \\
$X$ & Random variable \\
$[X=a]$ & $\{e \in \Omega \mid X(e) = a\}$ \\
$\mu_X$ & The distribution of the random variable $X$ \\
$(X_1, \dots, X_k)$ & Vector of random variables \\
$Ex[X]$ & Expectation of random variable $X$ \\
\bottomrule
\end{tabular}
\caption{Preliminaries}
\label{table:1}
\end{table}

A useful property of adjacency matrices is that $A^k(i,j)$ is exactly the number of distinct walks of length $k$ from vertex $v_i$ to vertex $v_j$. Moreover if the adjacency matrix $A$ has $A[i,i] = 1 \; \forall i$, then $A^k(i,j)$ is the number of walks of length at most $k$ from $v_i$ to $v_j$. See \cite{discreteHandbook} for a background.