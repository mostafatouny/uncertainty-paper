\textcolor{teal}{\textbf{broad intro to AV.}} Autonomous vehicles are a central pillar of many workflows in various industries. Yet, safety remains a critical bottleneck of its widespread adoption in public roads. The issue has social and regulatory implications as well. It is crucial to foster public trust, and address communication and transparency among stakeholders. A safe navigation relies on predictable environment alongside accurate sensors' data. Any unexpected environmental events or inaccuracies in sensors endanger people's lives. Understanding their sources, how they affect safety, and mitigating them is crucial for system's robustness and reliability.

\textcolor{teal}{\textbf{uncertainty sources and types.}} Those unpredictable elements include: weather conditions like rain, snow, fog, and strong winds; Dull lightning conditions; Terrains variations like slopes; Sensors noise as for example in LiDARs by signals distortion in fog; Dynamic movements of objects like vehicles and pedestrians whereby vehicles may take sudden lane changes or stops, and pedestrians may move in an unusual or unpredictable way \cite{alharbi2020global}. Technically, uncertainty could be in misclassifying objects, inaccurate positioning, unpredictable trajectories, or in sudden appearance of objects. The focus of this paper is in the first two categories of uncertainties.

\textcolor{teal}{\textbf{implications of uncertainty while neglecting or avoiding it.}} If a decision-making mechanism doesn't consider uncertainty, then this could clearly lead to unacceptable paths. A child may be hit because it is misclassified as a plastic bag \cite{tahir2024object}. A planner may hit an object due to distance measurements inaccuracies \cite{itkina2022uncertainty} whereby an object is reported to be more faraway than its actual position. If the mechanism aims to avoid uncertainty by being more conservative, then it is likely to complicate the path to avoid any potential threat. In some cases, a path avoiding all uncertainty may not be existing. Even if it exists, it needs to resolve more constraints incurring more computational time.

\textcolor{teal}{\textbf{literature approaches of reasoning under uncertainty.}} These uncertainties necessitate algorithms and modeling techniques capable of reasoning under uncertainty. Ongoing efforts in the literature include: Multi-sensor data integration \cite{vargas2021overview} where it addresses the limitations of individual sensors, such as Global Navigation Satellite Systems (GNSS) inaccuracies in urban canyons or LiDAR in adverse weather conditions, by merging data from various sensors; Motion planning \cite{tang2022driving} where new developments are in path density adjustments and the integration of hierarchical motion planning methods; Risk assessment \cite{wang2021risk} where a risk index is estimated based on various indicators like visibility and the presence of obstructions, allowing for dynamic updates in decision-making strategies; Probabilistic optimization frameworks \cite{han2022non} where the uncertainty of environmental conditions and obstacles are modeled. Those enable algorithms to generate risk-bounded decisions minimizing the likelihood of undesirable actions, as in trajectory planning.

\textcolor{teal}{\textbf{challenges of literature approaches.}} In an environment, some factors are completely known with total confidence, while others are uncertain. Their unpredictability degrees could range from low to high in the same environment. In practice, optimizing for these multiple degrees of uncertainty often requires considering each distribution or scenario separately, solving each with a distinct solution. For example, in \cite{OnboardMitigation_2} the planner is designed to be conservative to handle perception errors. The bounding boxes of all objects, regardless of perception accuracy, are enlarged. Similarly in \cite{OnboardMitigation_3}, an estimation or assumption of the perception uncertainty is required. In \cite{OnboardMitigation_4} an uncertainty adaptive planner is designed to estimate the degree of uncertainty. Therefore, in order to accommodate different distributions, and environmental factors, in a single module, a hybrid approach is frequently employed, where specialized models or methods are combined to handle various smaller sub-problems. However, the hybrid approach introduces several drawbacks: Scalability issues arise \cite{scalabilityML} because the complexity of the solution increases as multiple specialized solutions are integrated, to account for more factors or constraints; Additional time and cost are incurred due to designing, implementing, testing, and maintaining multiple models. More efforts are required for the integration and maintainability of the system as a whole; Reduced interpretability \cite{UncertainSurvey}, as it is more difficult to analyze or explain decision-making processes, based on multiple specialized solutions working together. Tracing would require analyzing each specialized module and how their outcomes are combined to produce the end result; Moreover, The effectiveness of data-driven models heavily depends on the quality of available datasets, as poor or biased data won't result in a robust agent. So accommodating multiple distributions and scenarios shall raise the concern of finding quality datasets for the whole case-study in hand; The trade-off among multiple objectives and scenarios in hybrid solutions requires a careful tuning on behalf of the designer, as favoring one objective could tamper others.

\textcolor{teal}{\textbf{contribution.}} In this work, we tackle object misclassification and inaccurate positioning uncertainties. We adopt graph theoretic modeling of the environment, where the vehicle's decision is a path over the digraph, computed algorithmically. It is independent of any particular scenario, dataset, environment knowledge. We choose probabilistic expectation as a quantitative measure for the path planning in uncertain environments. It captures both the likelihood of objects classification alongside associated penalties of their priorities to the planner. For instance, it is natural to assign hitting pedestrians a greater penalty than hitting inanimate objects like a banner. The linearity of expectation allows us to compute the total expected risk of a path by summing uncertain zones' expectations the path navigates through. It aligns well with traditional shortest path algorithms, which evaluate path weights by summing the costs of the vertices they traverse. Our solution reduces computational time by treating differently uncertain zones and neutral zones devoid of uncertainty. A faster subroutine is used for the latter, so that expensive shortest path queries are exclusive on uncertain regions.

\textcolor{teal}{\textbf{evaluation.}} Algorithmic correctness is proved using graph theory and logic techniques. Empirical emulation is based on dropout sampling, whereby a decision is taken according to a probability distribution over the digraph, then we emulate the decision on a sampled digraph, corresponding to the environment. Emulations are on Carla open-source simulator for autonomous driving research.

\textcolor{teal}{\textbf{Carla's implementation.}} Carla's waypoints correspond to vertices, and probabilities correspond to vertices' weights. Implementation is by Carla's built-in basic agent module \cite{py-carla}. Carla's built-in path planning modules were not used as more waypoints are needed within two-adjacent road segments. A simple Python script is written to construct a digraph from Carla's waypoints.
