We start with basic graph-theory and probability notation. We recall some useful properties of graphs' adjacency matrices. In Section \ref{Modeling}, we illustrate how graph-theoretic notions and vertices weights do capture the external uncertain environment of autonomous vehicles. We prove the equivalence of some scenarios with respect to probabilistic expectation. In Section \ref{Algorithm} we illustrate our framework from a higher-level view, showing how uncertain subgraphs are connected to neutral subgraphs. Then we give more details of subroutines. Namely, reducing vertex weights to edge weights, and querying a neutral path. The section ends with dynamic updates whereby a path is adjusted without recomputing everything from scratch. In Section \ref{Evaluation}, the dropout sampling methodology is explained, and how is it applied on digraph traversal. Finally, Section \ref{Experimentation} shows Carla emulation testing results on selected scenarios.
